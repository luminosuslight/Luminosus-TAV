\documentclass{scrartcl}
% deutsche Silbentrennung
\usepackage[ngerman]{babel}
% wegen deutschen Umlauten
\usepackage[utf8]{inputenc}
\usepackage[T1]{fontenc}
\usepackage{lmodern}

\usepackage[colorlinks, linkcolor = black, citecolor = black, filecolor = black, urlcolor = blue]{hyperref} 

\title{TAV - Report}
\subtitle{Project: Luminosus}
\author{Johannes Schneider and Tim Henning}

\begin{document}

\maketitle

\setcounter{section}{-1}
\section{Application Selection}

\subsection{Introduction}

Extending the functionality of a lighting desk can be a difficult and time consuming task. To make it more easy to integrate and test new features the idea was to create a separate modular software platform that is connected via network to the lighting console. As a solution for this "Luminosus" was developed.

The main project development was done by Tim Henning while writing his bachelor thesis. He is the only developer of the project till now. For this V\&V project Johannes Schneider joins the project as a developer and tester.

The programming language used is C++ 11 with syntax additions by Qt. The user interface part uses the descriptive language QML that includes function definitions in JavaScript.

The application is intended to be used by end users. It runs as a desktop application on Windows and MacOS and as an app on Android and iOS.

The existing code base consists of ca. 40k LOC.

Available artifacts are the source code, documentation, changelog, issue tracker, manual and in addition the Bachelor thesis \textit{Entwicklung einer modularen Benutzeroberfl\"ache als zus\"atzliche Bedieneinheit einer Lichtkonsole} including a requirement analysis and a discussion of the architectural decisions.

No specific development paradigm was used.

Manual tests are performed and static code analysis is provided by the IDE.

\subsection{Learning from failures}

\paragraph{Issue \href{https://github.com/ETCLabs/LuminosusEosEdition/issues/2}{\#2} Connection to Eos stopped working:}

After a while the network connection of the software to the lighting stopped working. The user was not able to continue using the application for the show even after restarting everything.
While the fault is not yet known the issue is probably complexity related. Many different devices were involved and it is not clear if the problem was caused by this software a  different failing part in the system.


\paragraph{Issue \href{https://github.com/ETCLabs/LuminosusEosEdition/issues/3}{\#3} Faders not updated:}

The user described an issue where the fader position were not updated correctly after switching the page.
This issues can be categorized as preventable as the fault was a single missing line of code that could have been found by unit tests that cover all branches of the code.


\paragraph{Issue \href{https://github.com/ETCLabs/LuminosusEosEdition/issues/4}{\#4} Fullscreen on iOS not working:}

The issues states that the fullscreen button is not working correctly on the iOS platform. The failure is visible as too large UI elements that render the application unusable.
On the one hand the fault can be categorized as preventable since it could be detected by a simple unit test that compares the scale property before and after the operation.
On the other hand it is also a type of intelligent fault since the real problem was the pure existence of a fullscreen button on a mobile operating system.

\paragraph{Conclusion}
In general the introduction of unit tests would be a valuable addition to the existing code base. Furthermore integration tests for typical user scenarios would help to prevent some kind of faults. Last but not least for maintaining so many different platforms the user feedback is an important part in the process of V\&V.


\subsection{Five Basic V\&V Questions}

\paragraph{When do V\&V start?  When are they complete?}

Validation started right in the beginning. The project was validated on a daily basis  due to a lot of contact to the supervisor. The specification was adjusted very frequently to ensure that the features that were developed were suitable for the projects goal.

Verification began early, too, in the form of static code analysis provided by the IDE (Qt Creator) and CLang. The development of tests did not start till the begin of this V\&V project.

The process of V\&V does not end till the end of the lifetime of the product. In this case as long as the software is maintained.

\paragraph{What particular techniques should be applied during development?}

Static code analysis is very good applicable for C++ due to strict type declarations. 

Continuous Integration should be applied to the project since it will be used on various platforms. Furthermore automated unit testing will be applied during the semester.

Additionally profiling the memory consumption can help finding memory leaks and performance benchmarks make sure that the specification is met in terms of latency and responsibility.

In the end a formal verification of standardized components such as the network protocol implementations should be applied.

\paragraph{How can we asses the readiness of the product?}

The readiness is assessed by using the defined features of the specification and their current implementation status. Before a release is considered ready it should be at least tested manually in a typical use case scenario.

Furthermore open issues are a hint that the product is not yet ready.

\paragraph{How can we control the quality of successive releases?}

To ensure that successive releases do not introduce new faults a pre-release policy can be established to let a small group of beta users test the system before it is released to the public.

\paragraph{How can the development process itself be improved?}

Increasing emphasis on applying various V\&V techniques during further development will help to improve overall code quality. 

Furthermore the development process can be improved by introducing a second developer to the project. Thus pair programming and code reviews can be established to help finding flaws. In addition maintaining the documentation will lead to a better process.


\section{Analysis}

\subsection{Human-based Static Analysis}

\paragraph{Peer Review}

The peer review helped understanding the code because it was possible to discuss the reviewed code directly with the author.
We were able to improve the general code quality and comments by pointing out parts that were difficult to understand.

The previously known bugs were found and we could even find a case where the description of a method did not match its content.

A good peer review is organized a few days before so that everyone can get into the code. Therefore the discussion can start right away. It also helps to question even minor things so that the author can either explain his thoughts or comes to really think of it for the first time.

\paragraph{Tools}
Search for tool support to perform peer review in your project and evaluate. Which features are supported? Which features would be desirable, but are not supported?

A tool that is well suited for peer reviews is the review function of GitHub. It allows to mark persons that have to review i.e. a pull request before it can be merged. Comments can be made for each line of code individually. It can integrate various different tools to enhance the experience and incorporate automated SA, too.

\subsection{Automated Static Analysis}

\paragraph{Evaluation of Tools}

We used Clang Static Analyzer and CppCheck for this project.
Both tools are well suited for C++ code and are supported by the IDE (Qt Creator). The detected faults are intuitively shown next to the corresponding line in the code.

On the in hand Clang helped to improve the code by finding a possible nullpointer derefence caused by a missing return statement. On the other hand it reported hundreds of false alarms in its default configuration, i.e. increase of precision of floating point numbers such making it harder to detect the real faults.

Other than Clang, CppCheck found not initialized member variables. But even while it was designed not to show false positives many of them were detected due to bugs in CppCheck.

\paragraph{Type of Inaccuracy}
Investigate if the ASA tools you used for fault detection assume some
pessimistic, optimistic inaccuracy or both and explain the consequences.

TODO

\subsection{Interaction between Human-Based and Automated SA}

By using ASA a peer review can be prepared in order to point out potentially broken code parts. It can then be discussed if a detected fault is considered a false positive or not.

When using tools like SonarCube the faults get matched with their author. This enables each developer to learn from their mistakes in a more direct way. Additionally it also displays which kind of faults are often produced in a project. These faults can then be targeted by further education and more awareness.

The results of a human-based peer review can be used to improve the configuration of the automated static analysis tools. This can reduce the number of false positives which are caused by the coding guidelines of the individual project and thus are intended.
Depending on the use case the relevance of the faults is differently prioritized. In our specific case the implicit increase of precision of floating point numbers is not a problem at all. In contrast in more space critical applications this could lead to unpredictable behavior.

\end{document}
